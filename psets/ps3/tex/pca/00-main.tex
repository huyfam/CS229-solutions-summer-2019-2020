\item \points{10} {\bf PCA} 

In class, we showed that PCA finds the ``variance maximizing'' directions onto
which to project the data.  In this problem, we find another interpretation of PCA. 

Suppose we are given a set of points $\{x^{(1)},\ldots,x^{(\nexp)}\}$. Let us
assume that we have as usual preprocessed the data to have zero-mean and unit variance
in each coordinate.  For a given unit-length vector $u$, let $f_u(x)$ be the 
projection of point $x$ onto the direction given by $u$.  I.e., if 
${\cal V} = \{\alpha u : \alpha \in \Re\}$, then 
\[
f_u(x) = \arg \min_{v\in {\cal V}} ||x-v||^2.
\]
Show that the unit-length vector $u$ that minimizes the 
mean squared error between projected points and original points corresponds
to the first principal component for the data. I.e., show that
$$ \arg \min_{u:u^Tu=1} \sum_{i=1}^\nexp \|x^{(i)}-f_u(x^{(i)})\|_2^2 \ .$$
gives the first principal component.


{\bf Remark.} If we are asked to find a $k$-dimensional subspace onto which to
project the data so as to minimize the sum of squares distance between the
original data and their projections, then we should choose the $k$-dimensional
subspace spanned by the first $k$ principal components of the data.  This problem
shows that this result holds for the case of $k=1$.

\ifnum\solutions=1 {
  \begin{answer}
By definition, for each kernel $K_i$, there must be some feature map $\phi_i$ 
such that $K_i(x, z) = \langle \phi_i (x), \phi_i (z) \rangle$. Moreover, 
recall that a kernel function as a dot product should be symmetric and $K(x,x) 
= x^{T}x = \|x\|_{2}^{2} \ge 0$. We will use these facts for the rest of this 
problem. 
\begin{enumerate}
\item 
\begin{align}
	K(x, z) 
	&= K_{1}(x, z) + K_{2}(x, z) \\
	&= \langle \phi_{1}(x), \phi_{1}(z) \rangle + \langle \phi_{2}(x), 
	\phi_{2}(z) \rangle \\
	&= \left\langle \begin{bmatrix}
		\phi_{1}(x) \\
		\phi_{2}(x) \\
	\end{bmatrix}, 
	 \begin{bmatrix}
		\phi_{1}(z) \\
		\phi_{2}(z) \\
	\end{bmatrix} \right\rangle
\end{align}
Hence, $K(x,z)$ is a valid kernel.

\item This can not be a kernel. For example, let's have $K_{1}(x,z)=1, K_{2}(x,z)=2$ which are deterministic but valid kernels. Then $K(x,z) = 1-2 = -1 < 0$, which is definitely not a kernel.

\item This function can be written as $K(x,z) = a 
\langle\phi_1(x),\phi_1(z)\rangle = \langle\sqrt{a}\phi_1(x),\sqrt{a}\phi_1(z)\rangle$ 
and so is a valid kernel.

\item Let's take as a counter-example $K_1(x,z)=1$, then $K(x,z) = -a < 0$ which is an 
invalid kernel. 

\item We have as follows:
\begin{align}
	K(x,z)
	&= K_1(x,z) K_2(x,z) \\
	&= \left( \sum \limits_{i=1}^{p} \phi_1^{(i)}(x)\phi_1^{(i)}(z) \right) \left( \sum 
	\limits_{i=1}^{p} \phi_2^{(i)}(x)\phi_2^{(i)}(z) \right) \\
	&= \sum \limits_{i=1}^{p} \sum \limits_{j=1}^{p} \left(\phi_1^{(i)}(x) 
	\phi_2^{(i)}(x)\right) \left(\phi_1^{(i)}(z) \phi_2^{(i)}(z)\right) \\
	&= \sum \limits_{i=1}^{p} \sum \limits_{j=1}^{p} \phi_{ij}(x)\phi_{ij}(z) \\
	&= \left\langle
	\begin{bmatrix}
		\phi_{11}(x) \\
		\vdots \\
		\phi_{pp}(x)
	\end{bmatrix},
	\begin{bmatrix}
		\phi_{11}(z) \\
		\vdots \\
		\phi_{pp}(z)
	\end{bmatrix}
	\right\rangle
\end{align}
In (7), we defined $\phi_{ij}(\cdot) = \phi_1^{(i)}(\cdot)\phi_2^{(j)}(\cdot)$. This 
result implies a legitimate kernel.

\item Think of the real-valued function $f(\cdot)$ as a one-component vector, $K(x,z) = f(x)f(z) = \langle f(x),f(z) \rangle$ is a kernel.

\item $K(x,z) = K_3(\phi(x),\phi(z)) = \langle \phi_3(\phi(x)), \phi_3(\phi(z)) \rangle$ is a kernel.

\item Without the loss of generality, we can write $p(y) = a_0 + a_{1}y + \dots a_{k}y^k$, where coefficients $a_0, a_1, \dots, a_k > 0, k \in \mathbb{N}$. Then,
\begin{align}
	K(x,z) 
	&= p(K_1(x,z)) \\
	&= a_0 + a_{1}K_1(x,z) + \dots + a_{k}(K_1(x,z)^k)
\end{align}
Using the previous results (a), (c), (e), we observe that each term in the sum is actually a kernel, and hence their sum $K(x,z)$ is also a valid one. \\
\end{enumerate}
\end{answer}



















} \fi

  
