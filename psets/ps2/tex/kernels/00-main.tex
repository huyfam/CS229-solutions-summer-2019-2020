\item \points{18} {\bf Constructing kernels}

In class, we saw that by choosing a kernel $K(x,z) = \phi(x)^T\phi(z)$, we can
implicitly map data to a high dimensional space, and have a learning algorithm (e.g SVM or logistic regression)
work in that space. One way to generate kernels is to explicitly define the
mapping $\phi$ to a higher dimensional space, and then work out the
corresponding $K$.

However in this question we are interested in direct construction of kernels.
I.e., suppose we have a function $K(x,z)$ that we think gives an appropriate
similarity measure for our learning problem, and we are considering plugging
$K$ into the SVM as the kernel function. However for $K(x,z)$ to be a valid
kernel, it must correspond to an inner product in some higher dimensional space
resulting from some feature mapping $\phi$.  Mercer's theorem tells us that
$K(x,z)$ is a (Mercer) kernel if and only if for any finite set $\{x^{(1)},
\ldots, x^{(\nexp)}\}$, the square matrix $K \in \Re^{\nexp \times \nexp}$ whose entries
are given by $K_{ij} = K(x^{(i)},x^{(j)})$ is symmetric and positive
semidefinite. You can find more details about Mercer's theorem in the notes,
though the description above is sufficient for this problem.
%
In this question we are interested to see which operations preserve the validity of kernels. 
%

Let $K_1$, $K_2$ be kernels over $\Re^{\di} \times
\Re^{\di}$, let $a \in \Re^+$ be a positive real number, let $f : \Re^{\di} \mapsto
\Re$ be a real-valued function, let $\phi: \Re^{\di} \rightarrow \Re^\nf$ be a
function mapping from $\Re^{\di}$ to $\Re^\nf$, let $K_3$ be a kernel over $\Re^\nf
\times \Re^\nf$, and let $p(x)$ a polynomial over $x$ with \emph{positive}
coefficients.

For each of the functions $K$ below, state whether it is necessarily a
kernel.  If you think it is, prove it; if you think it isn't, give a
counter-example.

\begin{enumerate}

\item \subquestionpoints{1} $K(x,z) = K_1(x,z) + K_2(x,z)$
\item \subquestionpoints{1} $K(x,z) = K_1(x,z) - K_2(x,z)$
\item \subquestionpoints{1} $K(x,z) = a K_1(x,z)$
\item \subquestionpoints{1} $K(x,z) = -a K_1(x,z)$
\item \subquestionpoints{5} $K(x,z) = K_1(x,z)K_2(x,z)$
\item \subquestionpoints{3} $K(x,z) = f(x)f(z)$
\item \subquestionpoints{3} $K(x,z) = K_3(\phi(x),\phi(z))$
\item \subquestionpoints{3} $K(x,z) = p(K_1(x,z))$

\end{enumerate}

[\textbf{Hint:} For part (e), the answer is that $K$ \emph{is} indeed
a kernel. You still have to prove it, though.  (This one may be harder than the
rest.)  This result may also be useful for another part of the problem.]

\ifnum\solutions=1 {
  \begin{answer}
By definition, for each kernel $K_i$, there must be some feature map $\phi_i$ 
such that $K_i(x, z) = \langle \phi_i (x), \phi_i (z) \rangle$. Moreover, 
recall that a kernel function as a dot product should be symmetric and $K(x,x) 
= x^{T}x = \|x\|_{2}^{2} \ge 0$. We will use these facts for the rest of this 
problem. 
\begin{enumerate}
\item 
\begin{align}
	K(x, z) 
	&= K_{1}(x, z) + K_{2}(x, z) \\
	&= \langle \phi_{1}(x), \phi_{1}(z) \rangle + \langle \phi_{2}(x), 
	\phi_{2}(z) \rangle \\
	&= \left\langle \begin{bmatrix}
		\phi_{1}(x) \\
		\phi_{2}(x) \\
	\end{bmatrix}, 
	 \begin{bmatrix}
		\phi_{1}(z) \\
		\phi_{2}(z) \\
	\end{bmatrix} \right\rangle
\end{align}
Hence, $K(x,z)$ is a valid kernel.

\item This can not be a kernel. For example, let's have $K_{1}(x,z)=1, K_{2}(x,z)=2$ which are deterministic but valid kernels. Then $K(x,z) = 1-2 = -1 < 0$, which is definitely not a kernel.

\item This function can be written as $K(x,z) = a 
\langle\phi_1(x),\phi_1(z)\rangle = \langle\sqrt{a}\phi_1(x),\sqrt{a}\phi_1(z)\rangle$ 
and so is a valid kernel.

\item Let's take as a counter-example $K_1(x,z)=1$, then $K(x,z) = -a < 0$ which is an 
invalid kernel. 

\item We have as follows:
\begin{align}
	K(x,z)
	&= K_1(x,z) K_2(x,z) \\
	&= \left( \sum \limits_{i=1}^{p} \phi_1^{(i)}(x)\phi_1^{(i)}(z) \right) \left( \sum 
	\limits_{i=1}^{p} \phi_2^{(i)}(x)\phi_2^{(i)}(z) \right) \\
	&= \sum \limits_{i=1}^{p} \sum \limits_{j=1}^{p} \left(\phi_1^{(i)}(x) 
	\phi_2^{(i)}(x)\right) \left(\phi_1^{(i)}(z) \phi_2^{(i)}(z)\right) \\
	&= \sum \limits_{i=1}^{p} \sum \limits_{j=1}^{p} \phi_{ij}(x)\phi_{ij}(z) \\
	&= \left\langle
	\begin{bmatrix}
		\phi_{11}(x) \\
		\vdots \\
		\phi_{pp}(x)
	\end{bmatrix},
	\begin{bmatrix}
		\phi_{11}(z) \\
		\vdots \\
		\phi_{pp}(z)
	\end{bmatrix}
	\right\rangle
\end{align}
In (7), we defined $\phi_{ij}(\cdot) = \phi_1^{(i)}(\cdot)\phi_2^{(j)}(\cdot)$. This 
result implies a legitimate kernel.

\item Think of the real-valued function $f(\cdot)$ as a one-component vector, $K(x,z) = f(x)f(z) = \langle f(x),f(z) \rangle$ is a kernel.

\item $K(x,z) = K_3(\phi(x),\phi(z)) = \langle \phi_3(\phi(x)), \phi_3(\phi(z)) \rangle$ is a kernel.

\item Without the loss of generality, we can write $p(y) = a_0 + a_{1}y + \dots a_{k}y^k$, where coefficients $a_0, a_1, \dots, a_k > 0, k \in \mathbb{N}$. Then,
\begin{align}
	K(x,z) 
	&= p(K_1(x,z)) \\
	&= a_0 + a_{1}K_1(x,z) + \dots + a_{k}(K_1(x,z)^k)
\end{align}
Using the previous results (a), (c), (e), we observe that each term in the sum is actually a kernel, and hence their sum $K(x,z)$ is also a valid one. \\
\end{enumerate}
\end{answer}



















} \fi
